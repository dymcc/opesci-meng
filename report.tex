\documentclass[a4paper,12pt,twoside]{report}
\usepackage[left=2cm,right=2cm,top=2cm,bottom=3cm]{geometry}
%\geometry{a4} 
\include{preamble}

%% Language and font encodings
\usepackage[english]{babel}
\usepackage[utf8x]{inputenc}
\usepackage[T1]{fontenc}
\usepackage{parskip}

%% Useful packages
\usepackage{amsmath}
\usepackage{graphicx}
\usepackage[colorinlistoftodos]{todonotes}
\usepackage[colorlinks=true, allcolors=blue]{hyperref}
\renewcommand{\baselinestretch}{1.2}

\title{Applying the polyhedral model to tile loops in Devito}
\author{Dylan McCormick}

\begin{document}
\maketitle

\tableofcontents

%%% INTRODUCTION %%%
\chapter{Introduction}
There are many established optimizations which improve the run-time performance of program loops and have been in use for decades.
The theory of why these optimizations work is well-researched [uniformloop] and has been practically applied in modern optimizing compilers so
transformed code need not always be hand-written. Some transformations have a negative effect on the readability and maintainability of code by
increasing code size and changing statement ordering. This can make them undesirable or even infeasible to develop and maintain by hand, in spite
of their performance benefits, and so it is an attractive option to have a smart compiler or code generator make the transformations automatically so the
original computation description remains concise. Automatically identifying and taking advantage of opportunities for loop optimizations is a tremendous
challenge for an optimizing tool which has to ensure program correctness based on the information available in the source code. If the tool cannot guarantee
correctness or appreciate the logical implications of a piece of code then no transformation occurs, resulting in a potential performance opportunity cost.
In a more domain-specific context, code-generation tools which are aware of the nature of the computation described by their input are less
restricted and more informed when it comes to identifying and applying optimizations.

In scientific computing applications code with a high arithmetic-intensity is common, particularly where numerical methods are employed.
Such computations often make extensive use of loops and loop nests to iterate over multi-dimensional structures like grids, matrices or
polyhedra. The code inside these loops is typically where these computations spend most of their time [dragon cite], so there is much to be gained
in terms of performance and efficiency by studying improvements and optimizations that can be made to this type of code, especially when dealing
with real-world large-scale problems like seismic imaging.

Automated code generation is an increasingly important solution to the problem of writing high-performance code
for scientific applications. Decades of research into compiler optimizations and advanced computer architecture developments
have uncovered a number of techniques that can be applied in translating high-level problem descriptions into low-level source code
to achieve some form of performance increase. The complexity of modern systems and computational problems makes it increasingly 
difficult for a software engineer to produce maximally efficient code by hand even with the help of smart
compilers. This is additionally challenging for scientists and researchers whose specialization is not in software engineering. Tools
and frameworks for automatic generation of application-specific high-performance code from high-level or even symbolic input are a very
attractive option for improving performance of numerical computations.

% http://dl.acm.org/citation.cfm?id=109108

\section{Loop tiling}
One optimization that can improve the performance of a loop nest is known as `loop tiling', which improves data locality and parallelism.
Loops which have been tiled can sometimes be further optimized in ways that were previously impossible. For example, 
loop-invariant code motion (LICM) transformations on non-tiled code can be restricted by available memory when 'moving'
statements from within a loop with a high number of iterations. This limitation can be overcome when LICM is applied to tiled code.
Loop tiling is one of the primary optimizations researched during this project.

\section{Main contributions}
The contributions of this project are as follows;
\begin{itemize}
	\item{Identify an opportunity for loop-tiling optimizations to be applied by Devito}
	\item{Demonstrate performance increase of optimizations applied to automatically-generated loop nests}
	\item{Devise and implement features in Devito which drive polyhedral tools to automate generation of tiled code}
	\item{Analyse performance results produced by new features}
	\item{Increase run-time performance gains realized by Devito users}
\end{itemize}

\chapter{Background}

\section{Fundamentals}
\subsection*{Compilers}
A compiler is a piece of software which produces code in a target language by assembling and interpreting input code in a source language.
Typically, compilers are used to translate a high-level programming language which describes computation to a human developer [example] into 
a low-level programming language [example] which is understood by a computer. Due to the complexity of modern languages, systems and computer
architectures this is not a straightforward translation and compilers have a lot of freedom in how they chose to interpret, transform and analyze
their input in order to produce output that meets certain requirements. For example, a compiler might reorder certain instructions from the input
program to produce output code which executes faster than without the instruction reordering.
[Compilation process diagram]

Source-to-source compilers, transpilers etc. This project is centered on improving performance of code compiled to machine code.

\subsection*{Abstract Syntax Tree}
An Abstract Syntax Tree (AST) is a tree data structure which represents the abstract syntax of a piece of
code in a particular language. The syntax is abstract in that it does not necessarily contain all of the
syntactic elements present in the original code. For example: Grouping parentheses are implied by the
trees structure and comments do not affect the program's behaviour so are omitted. ASTs are used 
by compilers to store and reason about an internal representation of the source code.

The specification document for a programming language defines and links the semantic rules that describe 
how language components behave, to the syntactic ones that define their structure. 
By applying the syntactic rules to raw text input, the compiler can identify and classify different 
language components that appear in the source code and build an AST that represents their structure.
\\
Pic of AST
\\
Once the compiler has constructed an AST it can perform semantic analysis on the statements represented
by the tree and understand the computation expressed by the code. With knowledge of the syntax and 
semantics of the source code, the compiler can make modifications and optimizations where it can and produce 
target code which expresses the same computation as the original source code. An AST is a powerful
data structure for any software that generates and analyses code.

\subsection*{Loops}
A common construct in modern code, especially when implementing numerical methods, is the `loop'. Virtually every program spends most of its 
time executing loops [Dragon 531], so compiler designers who are interested in optimization dedicate much research to 
loop transformations and analyses in an effort to reduce their performance impact. The type of loop dealt with in this project is the
`for loop' which is typically structured as in the figure below [FOR LOOP FIG (Use AST?)]. A `for loop' has a `body' of statements which are repeatedly executed
while the `loop condition' holds. There are no special constraints on the types of statements that can feature in the body of a for loop so it
is possible and indeed very common to have a `for loop' present in the body of another `for loop'. Placement of loops within other loops in such a manner
	is called `nesting' and gives rise to the `loop nest' construct which refers to a group of nested loops. /// Perfect loop nest

\subsection*{Loop Tiling}
Loops have a dominant impact on the performance of most programs that make use of them, particularly if loops are nested.
Most of the time, programmers write loops as succinctly and with as few redundant iterations as possible, making it difficult
for optimizations to truncate iteration spaces while preserving program correctness. Typically, loop optimizations reshape the
way a loop nest traverses its iteration space to try and exploit \textit{data locality} in the underlying computer architecture which
manages the data used by the loop body. There are several common loop optimizations which transform a loop nest structure ie. through
swapping the order of loop headers to exploit data locality, however the primary optimization studied in this paper is known as
\textit{loop tiling}.

\textit{Loop tiling}, also known as \textit{strip mine and interchange} or \textit{loop blocking}, is an optimization that modifies 
a loop nest so that rather than repeatedly iterating completely through each dimension of the iteration space until all points are
visited, the loop nest iterates completely over `tiles' or `blocks' of the iteration space. Tiling transformations group points of
the original iteration space into blocks which promote data reuse as the smaller blocks fit into fast memory and reduce the time between
uses of the same memory locations. In addition to improving data locality, loop tiling can also create opportunities for parallelism not 
present in the original code.

Fig X compares the iteration space of a non-tiled loop nest with that of the same loop nest after the loop-tiling has been applied.

\subsection*{Iteration spaces}
[Dragon 788]
% http://dl.acm.org/citation.cfm?id=2458526 Split Tiling
When analysing and describing iterative computations, particularly with nested iterative components, it can be useful to consider 
all of the iterations described by the code as points in an 'iteration space'. Each point in an iteration space represents a
particular assignment of values to the loop indices. Iteration spaces provide a geometric description of loop nests which can be
transformed and visualised. By considering connections between points in an iteration space, the order of iteration or 
inter-iteration dependencies can also be described.

Take this simple code fragment which iterates over a two-dimensional array and sets its elements to 0
\\ Some code here \\
Its iteration space is a rectangle of points representing all the combinations of the loop-indices that are within bounds and
arrows indicate the direction of iteration.
\\ Iteration space \\

A more complex two-dimensional loop nest might perform a computation at each iteration which requires a value from an earlier
computation.
\\ Some code here \\
In this case, arrows between points indicate a dependence on the source point by the target point. Both of these iteration 
spaces are the same shape and size as the two loop nests have the same limits and dimensions, but the
dependencies created by the loop body in the second example mean that the two spaces cannot be manipulated in the same ways.

Iteration spaces are used throughout this paper to motivate and explain the research and results obtained.

\subsection*{Polyhedral model}
% "A Practical Automatic Polyhedral Parallelizer and Locality Optimizer"
The Polyhedral model is a powerful abstraction for reasoning about loop nests and loop transformations [cite] that builds on the
geometric representation of loop iterations described in an iteration space. Each iteration of a statement is viewed as an
integral point within a polyhedron that contains all iterations of the statement. With a polyhedron for each statement and an understanding
of the dependencies between statements, linear algebra and linear programming techniques can be applied to transform and scan
the iteration spaces they represent.

\subsection*{Stencils}
% http://dl.acm.org/citation.cfm?id=1413375
A computation which iterates over a grid and performs nearest-neighbour computations is known as a stencil. These are
particularly common when implementing code that deals with numerical methods and partial differential equations.
Each point in the grid is updated some weighted contributions from a subset of its neighbours [stencilcite]. Figure X (iteration
space fig 2) is an example of a stencil.

%%% CLOOG %%%
\section{CLooG}
CLooG (Chunky Loop Generator) is a free software which generates loops for scanning Z-polyhedra (visiting each integral
point found within a convex polyhedron in lexicographical order), and is also designed to be
the back-end to code generation tools that perform automatic parallelism. Its output is pseudo-code loops which visit
each integral point of a union of polyhedra in such a way as to minimize control overhead and produce efficient code.

\subsection*{Motivation}
The polyhedral model allows reasoning about and solving a wide range of problems related to program transformations,
in particular where loop nests are involved. When performing these transformations, code generation is usually the last
step after the abstract structure of the program has been analysed and mutated. There are often constraints on the size of the code
that can be generated to ensure readability for developers and practicality for users attempting to run the code on their machine,
however the most concise code is not always the most efficient, and some transformations may generate complicated loop bounds in an effort
to reduce code size. This type of code can be difficult for a compiler to optimize and for a CPU to schedule in an optimal way due to
poor control management. CLooG provides an interface for applying polyhedral reasoning techniques to generate code which is optimized for control
and within the user's requirements of size or complexity.

We considered CLooG as a candidate for the time-tiling back-end for Devito because of its simplicity in use and its incorporation into
other tools performing similar tasks as this project's goal. The ability to manually build an input file for a simple example of Devito's
output provided an excellent starting point for understanding how both tools work and for obtaining initial results.
\subsection*{Implementation}
CLooG provides a command-line interface which makes use of specially formatted input files (primarily composed of matrices representing
the sets of inequalities defining an iteration space) to understand the loop nest structure as well as a C library that has structures and
functions that permit programmatic configuration and execution of the tool.
Although CLooG is frequently used for making loop-nests parallelizable or more control efficient, it is not concerned with the nature of the
code found within a loop (apart from other loops), and makes no assumptions whatsoever about dependencies between statements.
Statements in CLooG are represented abstractly by inequalities that define upon which iterations they are executed and other inequalities
which encode relative ordering between statements.

Once it has an understanding of the loop nest being transformed, and any additional constraints on statements imposed by the user
CLooG projects the polyhedra onto one dimension, separates the projection into distinct polyhedra sorted by lexicographic order and 
recursively generates loop nests for scanning the polyhedra. MORE ON THIS

\subsection*{Limitations}
CLooG is a generalised tool usually used as a back-end to a more specialised tool such as PLUTO or PrimeTile [citations] which
apply polyhedral theory to transform a given loop nest to enhance parallelism or data locality and delegate cloog to generate
efficient code to scan the new polyhedron. CLooG itself however, is not automated in any way and must be driven with a problem
specification that includes inequalities defining your iteration space, and 'scattering functions' which dictate the scheduling
of statements. This makes standalone usage of CLooG somewhat restrictive, but integration into Devito's code generation pipeline
very attractive. An additional technical constraint of CLooG is that it provides a command-line interface and C libraries only,
limiting its ability to interface with other languages such as Python which is common in scientific computing applications.

%%% PLUTO %%%
\section{Pluto}
PLuTo is an open-source tool for automatic polyhedral parallelization and locality optimization. Using CLooG as a code generation
back-end, PLuTo applies analytical model-driven transformations in the polyhedral model [cite] to improve the performance
of regular programs containing loop nests. One of the primary features of PLuTo is its automatic exploration of the space
of potential program transformations to identify and apply the most effective loop tiling.

\subsection*{Motivation}
Like CLooG, PLuTo is motivated by the powerful abstraction provided by the Polyhedral model when reasoning about and optimizing loop
nests. However PLuTo is more specifically concerned with how loop tiling can improve parallelism and data locality, especially
for applications run on massively parallel architectures like those frequently seen in scientific computing applications. The framework
takes a step further yet by applying analytical methods to intelligently find optimal transformations and tilings with code dependencies in mind.
The tool's design is indicative of its intended usage in optimizing existing code, possibly developed before a tool as capable as PLuTo was developed.

\subsection*{Implementation}
PLuTo operates as a source-to-source optimization tool, taking as input a sequence of nested loops in source-code format
and producing a transformed sequences of loops in source code. The research and behind PLuTo is on automated transformations,
not dependence analysis or source-code parsing, so it delegates the task of interpreting input and determining dependencies
to another tool in the polyhedral space, LooPo. The transformation framework implemented in PLuTo takes polyhedral domains representing the iteration
spaces of the original program, as well as dependence polyhedra which encode the dependence structure of the input source-code.

The transformation framework within PLuTo makes use of a bounding cost function to reason about dependences and iteration domains
geometrically and to provide a target to minimise when identifying optimal transformations. The cost function can be formulated
as a system of Integer Linear Programming (ILP) problems which encode legality and cost constraints and can be solved by the Simplex
algorithm. The framework iteratively augments and solves ILP problems to find multiple independent solutions for each statement. With
these solutions, PLuTo is able to construct matrices representing the transformed domains and statement dependencies which can be
given as input to CLooG for code generation.

\subsection*{Limitations}
PLuTo is a powerful tool that makes many informed analyses to drive its automated search for optimal transformations, and its integration
with other polyhedral tools make it a well-documented practical tool for users looking to apply source-to-source loop transformations. It
delivers a more specialised, automated solution than using CLooG alone, and allows with existing code to perform optimizations without having to 
analyse the loop and dependence structure of their program. However
PLuTo's generality and source-to-source nature make it unsuitable for direct application in the Devito pipline. With the specific structure and
nature of code generated by Devito, assumptions can be made that greatly reduce the work a tool like PLuTo would need to do in determining
dependencies and optimal transformations, but the general approach PLuTo takes prevents these assumptions from being made. The source-code level
is also a low-level representation that restricts what can be done by Devito after the transformations take place and would require premature
generation of source code to feed as input to PLuTo.
% https://pdfs.semanticscholar.org/d94d/d6f9ef3e15f3737738ad840647c9d23055cc.pdf
% http://drona.csa.iisc.ac.in/~uday//publications/uday-cc08.pdf

%%% Devito %%%
\chapter{Devito}
Devito is a Domain-specific Language and code generation framework for producing highly-optimised
code for finite-difference computations. It uses user-generated high-level symbolic problem descriptions to automatically
generate and optimise high-performance parallel C code for a range of computer architectures.
% https://arxiv.org/pdf/1605.06381.pdf
\subsection*{Motivation}
Modern high-performance computing applications need to make use of the latest developments in computer architecture to produce truly
powerful solutions. As modern technologies become more powerful they also become more complex with massively parallel systems like
GPGPU (General Purpose Graphics Processing Unit) and advanced architectural capabilities such as vectorization providing opportunities
to improve program performance at the cost of increased implementation complexity. For scientists looking to leverage performant
technologies in their programs the Python programming language is a common choice because of its natural form of expression,
large collection of open-source packages that allow interaction with some of these advanced architectures and its ease of use when compared with traditional languages like C.
However, as an interpreted language Python is not suited for direct use in HPC applications, and currently available methods for reducing the interpreter overhead introduce
additional complexity and are still not as efficient as straight C code. 

Devito aims to combine the ease and expressiveness of Python, with the 
power and optimizations available to C code. By using a domain-specific language embedded in a python package, Devito can provide a specialized interface
for describing numerical computations and perform code generation with aggressive optimizations to produce faster programs that require less HPC expertise
from the developer.

\subsection*{Design}
The high-level architecture of Devito can be broken into four steps;
\begin{itemize}
    \item Create Devito data objects which associate SymPy function symbols with user data
    \item Build symbolic stencil equations using the created data objects
    \item Build Devito Operator object using symbolic equations
    \item Instruct the Devito Operator to generate low-level optimized code applied to user data
    \item Compile using user-defined compiler settings
\end{itemize}
One of the key benefits of using Devito is the automatic generation of abstract array accesses upon creation of the Operator object. This 
creates an intermediate representation (IR) of the stencil which is then optimized and translated into C when triggered by the user using the Operator
in a process called 'propagation'. It is during this propagation stage that loops are generated, and so this is the point in the process where
Devito will analyse the IR and delegate to CLooG to tile the structure being generated. The objects that make up the IR allow
Devito to programmatically reason about the loop and dependence structure of the kernel which will form the primary input for CLooG's transformation.

\subsection*{Implementation}
Devito takes its form as a Python library that provides an API for users to create and manage objects that define and configure all of the components
in the code generation pipeline. To make the most of the native Python environment and the practical benefits it brings, Devito does not define its 
own high-level DSL in the traditional sense, instead it makes use of and extends the powerful SymPy Python library for symbolic mathematics. 
The familiar abstraction provided by a native python library allows users to separate implementation and optimization concerns from the numerical computation at hand.
\subsubsection*{Loop generation}



%%% PROJECT PLAN - Pathway to success %%%
\chapter{Project Plan}
Need Timeline!
\begin{itemize}
    \item Research polyhedral transformation and code generation tools
        \begin{itemize}
            \item This step is currently on-going. Having researched and experimented with the tools mentioned in
                this report, we have identified CLooG as a good starting point for performing the experiments needed
                to evaluate our plan.
        \end{itemize}
    \item Manually drive a tool to produce time-tiled code from a simple Devito-generated kernel
        \begin{itemize}
            \item Configuring and running CLooG at this stage has been an exercise in better understanding the polyhedral model,
                loop tiling and the code-generation interface CLooG provides. At present, we have successfully created a basic input file
                which describes a simple Devito kernel and run CLooG with it, however additional input constraints need to be understood and implemented
                before the output from these runs is interesting.
            \item To put Devito kernels in a state where they are ready to be transformed by CLooG, a feature was added to Devito's loop generation process
                which strip-mines the outer time loop in preparation for the tiling body transformations we intend for CLooG to make. This step was required as CLooG
                transformations cannot add extra dimensionality to a loop nest.
        \end{itemize}
    \item Perform performance experiments on tiled code produced by CLooG.
        \begin{itemize}
            \item With correctly tiled loop nests generated by manually-configured CLooG runs, we can run the code and compare its performance with an equivalent 
                non-tiled kernel. This will give an idea of how well the tool has been configured and what kind of performance gain we should expect to see when
                time-tiling is fully implemented.
        \end{itemize}
    \item Research implementation specifics to devise a workflow for driving CLooG automatically with Devito and feeding the tiled output back into the Devito pipeline.
    \item Implement CLooG as a back-end for automatic tiled code generation in Devito.
    \item Evaluate performance change in tiled Devito-generated kernels compared to non-tiled equivalents.
    \item If the transformations are successful in the way we expect, investigate the extent to which the transformations enable previously infeasible LICM transformations
        in Devito kernels and evaluate the performance benefit for any such LICM optimizations.
    \item Finalize implementation details and experiment results.
    \item Produce final write-up and presentation of results.
\end{itemize}

\chapter{Evaluation Plan}
This project specifically aims to improve performance of code generated by Devito, and evaluating the benefit of the changes made
as a result of research and experiments is crucial to determining the success of the project. Based on an understanding of the theory
behind loop transformations and of the behaviour of Devito kernels, there is an intuitive expectation that time-tiling in Devito will 
have a measurable improvement on kernel run time.
\begin{itemize}
    \item Once tiled code has been generated by CLooG with manual configuration experiments can begin on the performance
        of the tiled code. To determine the result of the program transformations, the non-tiled and tiled code should be independently
        and sequentially executed on hardware similar to that on which Devito kernels are usually run. Technical considerations must be made
        to minimise the influence of external factors (e.g. Operating System actions or other running processes) and provide a controlled testing
        environment. Multiple Devito kernels should be tested in this way and preference should be given to kernels with high iteration counts
        so that differences in execution time are more pronounced. Based on the results obtained at this stage, changes can be made to the configuration
        of CLooG as corrections or experiments with different tiling parameters. If the results indicate performance improvements then the project can advance
        to the implementation stage.
    \item Before implementation can fully begin, it is important that the design chosen for implementing a code generation back end to Devito be considered carefully.
        CLooG does not provide a native Python API and so with a project timescale in mind, a plausible implementation plan should be made.
    \item After the implementation stage, tests should be performed to ensure the numerical correctness of generated kernels and that transformations are
        being applied as expected.
    \item If no implementation bugs remain, performance testing can begin once again. With support in Devito for automated tiled-loop generation, evaluation at this
        stage should proceed in a large scale automated fashion. A set of problem descriptions that represent the different types of kernel Devito can produce is
        important to ensure that performance changes are understood across all applications of Devito. In particular, if for any reason a kernel experiences a slowdown
        then additional experiments must be performed to determine if the code can benefit from being transformed and if not, changes must be made in Devito to make sure
        it never generates sub-optimal code.
    \item When evaluating performance, several metrics should be considered;
        \begin{itemize}
            \item Devito run time when generating transformed programs
            \item Overall kernel run time
            \item Floating point operations per second (FLOPs)
            \item Cache performance (misses, hits)
            \item Compilation time and compiled size
            \item Memory usage
        \end{itemize}
    \item The primary criterion for success is decreased running time of Devito kernels, and this improvement should follow from improved cache performance 
        (fewer misses) and increased FLOPs.
    \item Since Devito is a user-facing tool, feedback should also be obtained regarding the applicability and usability of the new features.
\end{itemize}

%%% CITATIONS %%%
\chapter{Citations}
\bibliographystyle{alpha}
\bibliography{sample}

\end{document}
